\documentclass[aspectratio=43,11pt,xcolor={dvipsnames}]{beamer}
\usetheme{Madrid}
\usecolortheme{seahorse}
\usefonttheme{serif}
\usepackage[utf8]{inputenc}
\usepackage{amsmath}
\usepackage{amsfonts}
\usepackage{amssymb}
\usepackage{algorithm}
\usepackage{algpseudocode}
\hypersetup{pdfpagemode=FullScreen}
\usepackage{calc}
\usepackage[normalem]{ulem}% to striketrhourhg text
\usepackage{subcaption}
\usepackage{caption}
\captionsetup[figure]{labelformat=empty}% we don't need 'Figure' lable
\usepackage{smartdiagram}
\usesmartdiagramlibrary{additions}

\usepackage[style=verbose,backend=bibtex]{biblatex}
\renewcommand*{\bibfont}{\tiny}%font size for reference slide
\addbibresource{bibfile.bib}

\setbeamertemplate{navigation symbols}{}
\renewcommand{\footnotesize}{\tiny}

\title[AIR 2017]{Robotic cloth manipulation for clothing assistance task using Dynamic Movement Primitives}
\author[Ravi P. Joshi]{Ravi P. Joshi$^a$, Nishanth Koganti$^{a,b}$, and Tomohiro Shibata$^a$}
\institute[]{\tiny{$^a$Graduate School of Life Science and Systems Engineering, Kyushu Institute of Technology, Kitakyushu, Japan\\$^b$Graduate School of Information Science, Nara Institute of Science and Technology, Nara, Japan}}

\date[]{{June 29, 2017}}

% all the graphics are placed inside images folder
\graphicspath{{./images/}}

\titlegraphic{\includegraphics[scale=0.08]{kyutech} \hskip1cm \includegraphics[scale=0.08]{naist}}

%src: https://tex.stackexchange.com/a/248147
\defbeamertemplate{title page}{noinstitute}[1][]
{
  \vbox{}
  \vfill
  \begingroup
    \centering
    \vskip3em\par
    \begin{beamercolorbox}[sep=4pt,center,#1]{title}
      \usebeamerfont{title}\inserttitle\par%
      \ifx\insertsubtitle\@empty%
      \else%
        \vskip0.1em%
        {\usebeamerfont{subtitle}\usebeamercolor[fg]{subtitle}\insertsubtitle\par}%
      \fi%     
    \end{beamercolorbox}%
    \vskip0em\par
    \begin{beamercolorbox}[sep=8pt,center,#1]{author}
      \usebeamerfont{author}\insertauthor
    \end{beamercolorbox}
    \begin{beamercolorbox}[sep=8pt,center,#1]{institute}
      \usebeamerfont{date}\insertinstitute
    \end{beamercolorbox}\vskip0.1em
    \begin{beamercolorbox}[sep=8pt,center,#1]{date}
      \usebeamerfont{date}\insertdate
    \end{beamercolorbox}\vskip0.1em
    {\usebeamercolor[fg]{titlegraphic}\inserttitlegraphic\par}
  \endgroup
  \vfill
}

\makeatletter
\setbeamertemplate{title page}[noinstitute][colsep=-4bp,rounded=true,shadow=\beamer@themerounded@shadow]
\makeatother

\begin{document}
{
	\usebackgroundtemplate{\includegraphics[width=\paperwidth]{background}}
	\setbeamertemplate{footline}{} 
	\begin{frame}
		\titlepage
	\end{frame}
}
\addtocounter{framenumber}{-1}

%\begin{frame}[noframenumbering]
%	\titlepage
%\end{frame}

\begin{frame}[noframenumbering]{Outline}
	\tableofcontents
\end{frame}

\section{Introduction}
\begin{frame}{Introduction}
	\linespread{1.5}
	\begin{columns}[t]
		\begin{column}{0.65\textwidth}
			\begin{itemize}
				\item Clothing assistance is a basic and important assistance activity in the daily life of the elderly and disabled people
				\item Need of robotic clothing assistance is growing
			\end{itemize}
		\end{column}
						
		\begin{column}{0.35\textwidth}
			\begin{figure}
				\includegraphics[width=0.8\textwidth]{robotic_clothing_assistance}
			\end{figure}
		\end{column}
	\end{columns}
	
	
	\vskip 1em
	\begin{block}{Major challenges involved}
		\begin{itemize}
			\item Close interaction of the robot with non-rigid clothing article
			\item Safe human-robot interaction
			\item Accurate estimation of human-cloth relationship
		\end{itemize}
	\end{block}
\end{frame}

\section{Related Works}
\begin{frame}{Related Works}
			
	\vskip -1em
	\begin{columns}[t]
		\begin{column}{0.5\textwidth}
			Towner \textit{et al.}\footnotemark, ~Identifying and manipulating clothing article by dual-arm robot
			{\scriptsize
				\begin{itemize}
					\item[\color{green}{\checkmark}] Used Hidden Markov Model for tracking
					\item[\color{green}{\checkmark}] Triangulated mesh model for simulating clothing article
					\item[\color{red}{$\times$}] Highly depends on simulated contour information.
				\end{itemize}
			}
			\vskip 0.5em
			\centering{
				\includegraphics[height=2.5cm]{towner_2011}
			}
		\end{column}
						
		\begin{column}{0.5\textwidth}
			Tamei \textit{et al.}\footnotemark, ~Clothing assistance with dual-arm robot
			{\scriptsize
				\begin{itemize}
					\item[\color{green}{\checkmark}] Used Reinforcement
					      learning (RL)
					\item[\color{green}{\checkmark}] Topology coordinates for human and cloth extremities relationship
					\item[\color{red}{$\times$}] Limited generalization capability for new postures
				\end{itemize}
			}
			\vskip 0.5em
			\centering{
				\includegraphics[height=2.5cm]{tamei_2011}
			}
		\end{column}
	\end{columns}
			
	\addtocounter{footnote}{-1}
	\footcitetext{cusumano2011bringing}
	\stepcounter{footnote}
	\footcitetext{tamei2011reinforcement}
\end{frame}

\section{Dynamic Movement Primitives}
\begin{frame}{Dynamic Movement Primitives (DMP)}
	%\linespread{1.2}
			
	\begin{exampleblock}{DMP in a nutshell}
		\begin{itemize}
			\item It is used for generating a control signal to guide the real system\footnotemark
			\item It can represent \textit{nonlinear} motion with a set of differential equations
		\end{itemize}
	\end{exampleblock}
			
			
	%	We start with point attractor dynamics\footcite{ijspeert2002movement} \hskip1em $\ddot{y} = \alpha_y ( \beta_y (g - y) - \dot{y})$
			
	%	\vskip -1em
	%	\begin{equation}
	%		\ddot{y} = \alpha_y ( \beta_y (g - y) - \dot{y})
	%		\label{eq:attractor}
	%	\end{equation}
	%	\vskip -0.5em
			
	%	\vskip 1em
	%	Now add a forcing term $f$ on eq(\ref{eq:attractor}) that will let us to modify this trajectory
	The system is defined as
	\vskip -1em
	\begin{equation}
		\ddot{y} = \alpha_y ( \beta_y (g - y) - \dot{y}) + f
		\label{eq:attractor_with_f}
	\end{equation}
	%\vskip -0.5em
	{\small
		where:
		\vskip -0.6em
		\begin{itemize}
			\itemsep-0.2em
			\item $y$ is system state and $g$ is goal state
			\item $\alpha$ and $\beta$ are gain terms
			\item $f$ is nonlinear function defined over time
		\end{itemize}
	}
			
	\begin{exampleblock}{}
		$f$ is a function of \textit{canonical system}, denoted by $x$ as $\dot{x} = -\alpha_x x$
	\end{exampleblock}
			
	\footcitetext{schaal2006dynamic}
\end{frame}


\begin{frame}{Forcing function $f$}
	$f$ is defined as
	\begin{equation}
		f(x,g) = \frac{\Sigma_{i=1}^N \psi_i w_i}{\Sigma_{i=1}^N \psi_i} x(g - y_0)
		\label{eq:cs_with_f}
	\end{equation}
			
	where:
	\begin{itemize}
		\item $y_0$ is the initial state of the system
		\item $w_i$ is a weighting for a given basis function $\psi_i$
		\item $\psi_i = \textrm{exp}\left( -h_i \left( x - c_i\right)^2 \right)$ is Gaussian with mean $c_i$ and variance $h_i$
	\end{itemize}
			
	\vskip -1em
	\begin{columns}
		\begin{column}{0.5\textwidth}
			\begin{figure}
				\includegraphics[height=2.1cm]{psi_activation}
				\vskip -0.3em
				\caption{$\psi$ Activation}
			\end{figure}
		\end{column}
						
		\begin{column}{0.5\textwidth}
			\begin{figure}
				\includegraphics[height=2.1cm]{weighted_summation}
				\vskip -0.3em
				\caption{Weighted Summation}
			\end{figure}
		\end{column}
	\end{columns}
\end{frame}

\begin{frame}{Imitating a desired path}
	\linespread{1.4}
	The desired forcing term $f$ which affects the system acceleration, is written as
	\vskip -1.5em
	\begin{equation}
		\textbf{f}_d = \ddot{\textbf{y}}_d - \alpha_y ( \beta_y (g - \textbf{y}) - \dot{\textbf{y}})
		\label{eq:imitate_f}
	\end{equation}
	\vskip -0.5em
	{\small
		where 
		\vskip -1em
		\begin{itemize}
			\item $\textbf{y}_d$ is desired trajectory, given by $\ddot{\textbf{y}}_d = \frac{\partial}{\partial t} \dot{\textbf{y}}_d = \frac{\partial}{\partial t} \frac{\partial}{\partial t} \textbf{y}_d$
		\end{itemize}
	}
	\vskip-0.5em
	\begin{exampleblock}{}
		Choose the weights over the basis functions i.e., minimize\footnotemark
		\vskip-2em
		\begin{equation}
			\Sigma_t \psi_i(t) \left[ f_d(t) - w_i \left\lbrace x(t) (g - y_0)\right\rbrace\right]^2
			\label{eq:minize_fd}
		\end{equation}
	\end{exampleblock}
	\vskip-0.5em
	\begin{figure}
		\includegraphics[width=0.8\textwidth]{imitate_path}
	\end{figure}
	\vskip-1.5em
	\footcitetext{schaal2002scalable}
\end{frame}

\section{Setup and Experiment}
\begin{frame}{Workflow of \textit{Robotic cloth manipulation} task}
	\vskip -0.5em
	\begin{figure}
		\includegraphics[width=0.9\textwidth]{flowchart_beamer}
	\end{figure}
\end{frame}

\begin{frame}{Setup}
	\begin{figure}
		\includegraphics[width=\textwidth]{setup}
	\end{figure}
\end{frame}

\section{Results}
\begin{frame}{Results}
	\begin{columns}[b]
		\begin{column}{0.5\textwidth}
			\begin{figure}
				\includegraphics[width=\textwidth]{various_posture}
				\caption{Old \& modified posture of mannequin}
			\end{figure}
		\end{column}
						
		\begin{column}{0.5\textwidth}
			\begin{figure}
				\includegraphics[width=\textwidth]{trajectory}
				\caption{Left arm trajectories of Baxter Robot}
			\end{figure}
		\end{column}
	\end{columns}
			
	\vskip 1em
	\centering{
		\href{run:./videos/video.mp4}{\beamerbutton{Video demonstration}}
	}
\end{frame}

\section{Conclusion and Discussion}
\begin{frame}{Conclusion and Discussion}
	\linespread{1.5}
	
	\begin{itemize}
		\item Baxter APIs\footcite{fitzgerald2013developing} are used to get the end-effector forces. Raw forces are found noisy in nature.
		\item Result shows that DMPs are able to generalize the movement trajectory
		\item Proposed failure detection method by using force information can detect failures
		\item DMP should incorporate orientation information as well
	\end{itemize}
\end{frame}

\section{Future work}
\begin{frame}{Future work}
	\linespread{1.5}
	
	\begin{itemize}
		\item Make approach more robust by using combination of visual and force information
		\item Need for designing an adaptive controller
		      \begin{itemize}
		      	\item For real-time tracking of mannequin
		      	\item To adapt various failure scenarios
		      \end{itemize}
	\end{itemize}
		
	\begin{exampleblock}{Acknowledgments}
		This work was supported in part by the Grant-in-Aid for Scientific Research from Japan Society for the Promotion of Science (No. 16H01749).
	\end{exampleblock}
\end{frame}

\begin{frame}[noframenumbering]{References}
	\nocite{*}
	\hspace*{0.5cm}
	\begin{minipage}{\dimexpr\textwidth-1cm\relax}
		\printbibliography
	\end{minipage}
\end{frame}



\begin{frame}[noframenumbering]{The End}
	\begin{center}
		\Huge Thanks for your attention!
		\vskip 1em \huge Any questions?
		\vskip 2em \normalsize \url{www.ravijoshi.xyz}
	\end{center}
\end{frame}

\end{document}
