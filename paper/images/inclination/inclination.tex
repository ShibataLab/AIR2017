\documentclass{standalone}
\usepackage{graphicx}
\usepackage{tikz}
\usetikzlibrary{shapes.geometric, arrows}

\newcommand*{\BoxWidth}{4cm}%
\tikzset{
    to/.style     = {->},
    line/.style   = {line width=0.4mm, draw=blue},
    darrow/.style = {<->, line width=0.4mm, draw=blue},
    box/.style    = {draw, rectangle, minimum height = 1cm, minimum width = \BoxWidth, text centered, text width=\BoxWidth, ultra thin, draw=black, fill=blue!10, font=\large}
}

\pgfmathsetmacro{\ex}{0.85}
\pgfmathsetmacro{\ey}{0.60}

\begin{document}
\begin{tikzpicture}
	\node[anchor=south west,inner sep=0] (image) at (0,0,0) {\includegraphics[width=\linewidth]{mannequin}};
	\begin{scope}[x={(image.south east)},y={(image.north west)}]
		\iffalse
			%% Next four lines helps to locate the point needed by forming a grid
			\draw[help lines,xstep=.1,ystep=.1] (0,0) grid (1,1);
			\draw[help lines,xstep=.05,ystep=.05] (0,0) grid (1,1);
			\foreach \x in {0,1,...,9} {\node[anchor=north] at (\x/10,0) {0.\x}; }
			\foreach \y in {0,1,...,9} {\node[anchor=east] at (0,\y/10) {0.\y};}
		\fi
		\begin{scope}[x={(image.south east)},y={(image.north west)}]
			\draw[line] (0.30,0.60) -- (\ex,\ey);
			\draw[line] (0.20,0.25) -- (\ex,\ey);
			\draw[darrow](\ex,\ey) ++(-152:.2) arc (-152:-180:.2);
			\draw[to](0.40,0.65) node[box]{Angle of inclination} to [out=-90,in=200] (0.65,0.55);
		\end{scope}
	\end{scope}
\end{tikzpicture}
\end{document}
